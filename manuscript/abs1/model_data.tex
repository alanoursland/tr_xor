% !TeX root = ../main.tex
\section{Model \& Data}
\label{sec:abs1-model-data}

\subsection*{Dataset: Centered XOR}
The canonical XOR points \((0,0),(0,1),(1,0),(1,1)\) are translated to
\(\{-1,1\}^2\) so that \emph{each feature has zero mean}.  
This centering

\begin{enumerate*}[label=(\roman*)]
  \item removes the need for an explicit bias term in the analytic optimum,
  \item preserves rotational symmetry about the origin, and
  \item follows the common machine-learning practice of zero-mean inputs.
\end{enumerate*}

\begin{table}[h]
\centering
\caption{Centered XOR dataset used throughout the chapter.}
\label{tab:xor-data}
\begin{tabular}{@{}cc|c@{}}\toprule
$x_1$ & $x_2$ & Target $y$ \\ \midrule
$-1$ & $-1$ & $0$ \\
$-1$ & $\;\phantom{-}1$ & $1$ \\
$\;\phantom{-}1$ & $-1$ & $1$ \\
$\;\phantom{-}1$ & $\;\phantom{-}1$ & $0$ \\ \bottomrule
\end{tabular}
\end{table}

\subsection*{Model Architecture}
All experiments share the same \emph{single-unit} network shown in
Figure~\ref{fig:model-graph} and expressed analytically as
\begin{equation}
    \hat y(x)
    =\;
    \bigl\lvert\,w^{\mathsf T}x + b\,\bigr\rvert,
    \quad
    w \in \mathbb{R}^{2},\;
    b \in \mathbb{R}.
    \label{eq:abs1}
\end{equation}

\begin{figure}[h]
\centering
\[
    \text{Input }(2)
    \;\xrightarrow{\;\,w,b\,\;}
    \text{Linear}
    \;\xrightarrow{\;|\cdot|\;}
    \hat y
\]
\caption{Computational graph for the Abs1 model.}
\label{fig:model-graph}
\end{figure}

\paragraph{Loss function}
We use the mean-squared error
\begin{equation}
    \mathcal{L}(w,b)
    \;=\;
    \frac{1}{4} \sum_{i=1}^{4} 
    \bigl(\,\lvert w^{\mathsf T}x_i + b\rvert - y_i\bigr)^2,
    \label{eq:mse-loss}
\end{equation}
whose Hessian is the constant matrix \(H = 2I\) once a sign pattern is
fixed—an analyticity that allows an exact Newton step with learning-rate
\(\eta = 0.5\).

\paragraph{Analytic optimum and prototype surface}
Because the model contains only two weights and one bias, the mean-squared
error can be written in closed form.  Substituting the centered XOR points into
\eqref{eq:mse-loss} yields

\[
  L(w_1,w_2,b)\;=\;
  4b^2 + 4w_1^2 + 4w_2^2
  - 2\bigl|\,b-w_1+w_2\bigr|
  - 2\bigl|\,b+w_1-w_2\bigr| + 2.
  \tag{1}\label{eq:closed-loss}
\]

Minimising \eqref{eq:closed-loss} is straightforward:

\begin{enumerate*}[label=(\roman*)]
  \item The gradient with respect to \(b\) vanishes only when
        \(b = 0\).
  \item Setting \(b=0\) reduces the two absolute-value terms to
        \(\lvert -w_1+w_2\rvert\) and \(\lvert w_1-w_2\rvert\),
        forcing \(w_1 = -w_2\).
  \item Writing \(w_1 = \alpha\) then makes both absolute terms
        \(\lvert -2\alpha\rvert\).  Minimising the quadratic part
        \(4\alpha^2 + 4\alpha^2\) subject to
        \(\lvert -2\alpha\rvert = 1\) gives \(\alpha = \pm\tfrac12\).
\end{enumerate*}

Hence the global minima are the two sign-symmetric parameter sets

\[
  w^\star = \Bigl(\frac12,\,-\frac12\Bigr),\quad b^\star = 0,
  \qquad\text{or}\qquad
  \bigl(-w^\star,\,-b^\star\bigr).
\]

\paragraph{Geometric interpretation}
With \(w_1=-w_2\) and \(b=0\), the pre-activation
\(f(x)=\tfrac12(x_1-x_2)\) defines the line \(x_1=x_2\).
This line intersects the two \textbf{False} inputs
\(({-}1,{-}1)\) and \((1,1)\); the network therefore assigns them an output
of~0.  
The remaining \textbf{True} inputs lie at a Euclidean distance
\(\sqrt2\) from the line, giving them output~1 and driving the loss to zero.

Crucially, the \emph{intersection itself} is what the model learns:  
the defining feature is not a high-magnitude activation but the exact location
where the affine form \(w^{\mathsf T}x+b\) vanishes.
In Prototype-Surface Learning terms, the locus \(f(x)=0\) is the
\emph{prototype surface} for the False class; parallel level sets
\(f(x)=\pm\sqrt2\) through the True points form additional, implicit
surfaces whose distance encodes class membership.
Because those parallel surfaces never attain the reference value of zero,
they are harder to isolate geometrically, yet they follow directly from the
same learned parameters.

Empirically, every successful run in the experiments that follow converges to
one of the two sign-symmetric optima derived above, confirming that the model
indeed learns by anchoring its surface to the False inputs and placing the
True inputs on parallel offsets—exactly as the theory predicts.
