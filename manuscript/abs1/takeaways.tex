\section{Conclusions}
\label{sec:abs1-conclusions}

\subsection*{Universal Success and Geometric Consistency}

This chapter demonstrates that the single absolute-value neuron architecture achieves remarkable robustness for XOR classification, with 100\% success across all initialization schemes and optimizer configurations tested. More significantly, Table~\ref{tab:abs1-weight-clusters} and Figure~\ref{fig:abs1-hyperplanes} reveal that regardless of weight scale or optimization strategy, training invariably produces the same geometric configuration:

\begin{enumerate}[label=(\alph*)]
    \item A single \textbf{distance pattern}: class-False points lie exactly on the learned hyperplane, class-True points are positioned at distance $\sqrt{2}$
    \item Two \textbf{weight clusters}: parameter vectors group into sign-symmetric optima $(\frac{1}{2},-\frac{1}{2},0)$ and $(-\frac{1}{2},\frac{1}{2},0)$
\end{enumerate}

These findings provide direct empirical validation of the analytical solution from Section~\ref{sec:abs1-model-data} and demonstrate the core principle of Prototype-Surface Learning: the neuron encodes class membership through the location of its zero-level set, not through the magnitude of its positive activations.

\subsection*{Fundamental Insights}

The experiments reveal three key principles that will guide subsequent investigations:

\paragraph{Speed-Destination Separation}
Initialization scale and optimizer choice affect only the optimization trajectory, never the final learned representation. Whether starting with tiny weights ($\sigma=0.1$) or large weights ($\sigma=4.0$), and whether using SGD or Adam, the prototype surface always anchors to the False class and positions the True class at the theoretically predicted distance.

\paragraph{Optimization Theory Validation}
The quadratic loss surface enables direct empirical testing of optimization theory. SGD with learning rate $\eta=0.5$ achieves the theoretical ideal by matching Newton steps exactly, converging in essentially one parameter update. This controlled validation demonstrates the value of analytical tractability for understanding optimization dynamics.

\paragraph{Geometric Robustness}
The consistent emergence of identical prototype surface structures across all experimental conditions suggests that these geometric relationships represent fundamental attractors in the learning dynamics rather than artifacts of specific training procedures.

\paragraph{Convergence Prediction}
Our analysis reveals hints of systematic relationships between initialization geometry (angle changes, norm ratios) and convergence speed, but these patterns require more sophisticated modeling to quantify precisely. Developing predictive metrics for training time based on geometric displacement could inform initialization strategies for complex architectures.

\paragraph{Learned Symmetry Challenge}
The next chapter decomposes the absolute value into two independent ReLU neurons: $|w \cdot x + b| = \text{ReLU}(w_0 \cdot x + b_0) + \text{ReLU}(w_1 \cdot x + b_1)$. This transition from hard-coded to learned symmetry will test whether the clean separation between optimization speed and learned geometry persists when neurons must coordinate rather than operate in isolation.
