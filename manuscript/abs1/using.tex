% !TeX root = ../main.tex
\section{Using This Chapter}
\label{sec:abs1-using}

This chapter is written to serve two kinds of readers at once:

\begin{enumerate}
    \item \textbf{The survey reader} who wants the headline results and their
          conceptual implications.
    \item \textbf{The detail-oriented reader} who wishes to trace every number
          back to its experimental source.
\end{enumerate}

\noindent
The text therefore mixes concise “mini-reports” with pointers to richer
artefacts placed in the appendix and in the digital supplement.%
\footnote{The supplement contains scripts and raw outputs that reproduce every
table and figure, but familiarity with them is \emph{not} required for
understanding the narrative here.}

\subsection*{Navigation Tips}

\begin{itemize}
    \item \textbf{Skim or dive}: Each numbered section opens with a boldface
          paragraph that summarises its main message; the rest can be read
          selectively.
    \item \textbf{Cross-references}: When a statement relies on theory developed
          elsewhere, we reference the relevant chapter or appendix section.  No
          background beyond what is cited is assumed.
    \item \textbf{Per-chapter bibliography}: References that appear in this
          chapter are listed immediately after the chapter (see the end of the
          document).  Citations from other chapters do not clutter this list.
\end{itemize}

\subsection*{Legend for Mini-Reports}

Throughout Sections~\ref{sec:abs1-init} and~\ref{sec:abs1-optim} you will
encounter standardized mini-reports.  Each follows the same template:

\begin{description}
  \item[Header] Experiment tag and one-sentence description.
  \item[Convergence Table]   Five-quantile summary of epochs required to reach
        an MSE below \(10^{-7}\).
  \item[Geometry Summary]  Count of distinct hyperplane clusters and average
        distance of class--1 points to the learned surface.
\end{description}

This uniform layout lets you compare variants at a glance, while fuller plots
and diagnostic statistics reside in the appendix.

\subsection*{Reproducibility Note}

All numerical results and figures in this chapter are generated
\emph{programmatically}.  The exact computational workflow—including data
generation, model definitions, and analysis scripts—is archived with a fixed
digital identifier and will build automatically on a standard \LaTeX\ tool
chain plus a recent Python environment.  Readers interested in re-running the
experiments may consult the short how-to guide provided in the digital
supplement; the main text remains free of implementation details.

\bigskip
\noindent
\textit{In short: browse the headlines, follow the cross-references as needed,
and consult the appendix or supplement only when you wish to dive deeper into
the numbers.  The chapter is designed to accommodate both quick intuition and
full reproducibility without forcing either mode on the reader.}
