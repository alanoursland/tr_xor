% !TeX root = ../main.tex
\section{Mirror Initialization Study}
\label{sec:relu1-mirror}

\subsection*{Motivation}
Because \(|z| = \operatorname{relu}(z) + \operatorname{relu}(-z)\), a
\emph{two-ReLU} network can in principle emulate the single-Abs model if
its two hidden weight vectors begin as perfect negatives of one another.
The \texttt{init\_mirror} routine therefore samples one weight-bias pair
from \(\mathcal N(0,1)\) and assigns its exact negation to the second
neuron, guaranteeing mirror symmetry from the first step.

% ------------------------------------------------------------------
\subsection*{Classification Accuracy}

\begin{table}[h]
\centering
\caption{Final accuracy across $1000$ mirrored initialisations.}
\label{tab:relu1-mirror-accuracy}
\begin{tabular}{lccccc}
\toprule
Accuracy & 0\,\% & 25\,\% & 50\,\% & 75\,\% & 100\,\% \\
\midrule
Runs & 0 & 0 & 16 & 0 & 984 \\
\bottomrule
\end{tabular}
\end{table}

Mirror seeding yields a \textbf{98.4\,\%} success rate-the highest of
all single-shot initialisation schemes.

% ------------------------------------------------------------------
\subsection*{Convergence Timing}

\begin{table}[h]
\centering
\caption{Epochs to $\mathcal L<10^{-7}$ for the 984 successful runs.}
\label{tab:relu1-mirror-epochs}
\begin{tabular}{lcccccc}
\toprule
Percentile & 0\,\% & 10\,\% & 25\,\% & 50\,\% & 75\,\% & 100\,\% \\ \midrule
Epochs & 6 & 39 & 62 & 96 & 138 & 316 \\
\bottomrule
\end{tabular}
\end{table}

Median runtime (96 epochs) beats every previous variant except the tiny
positive-leak activations.

% ------------------------------------------------------------------
\subsection*{Prototype-Surface Geometry}

\begin{description}[leftmargin=2em]
  \item[Distance clusters]  
        All 1968 hyperplanes from successful runs collapse to a single
        pattern, \((d_{0},d_{1})=(0.10,1.41)\); the prototype surface
        sits nearly on the False points and \(\sqrt2\) from the True
        points. :contentReference[oaicite:2]{index=2}
  \item[Weight clusters]  
        DBSCAN finds exactly \textbf{two} sign-symmetric clusters, each
        containing 984 weights whose centroids are \(\pm(0.54,-0.55)\). :contentReference[oaicite:3]{index=3}
  \item[Mirror symmetry]  
        Every successful run maintains a perfect mirror pair (cosine
        \(=-1\)). :contentReference[oaicite:4]{index=4}
\end{description}

% ------------------------------------------------------------------
\subsection*{Failure Analysis}
The remaining 16 runs all stall at 50 \% accuracy.  Hyperplane-angle
statistics show their initial mirrors are
\(\approx\!90^{\circ}\) from any optimum and never rotate far enough
before the companion plane minimises loss locally-a reprise of the
"perpendicular trap" seen earlier. :contentReference[oaicite:5]{index=5}

% ------------------------------------------------------------------
\paragraph{Discussion}
\begin{itemize}
  \item Mirrored weights almost eliminate dead-data and orientation
        variance in one shot, giving the best reliability-speed trade-off
        among static inits.
  \item Geometry is pristine: a single distance pattern, two perfect
        weight clusters, and universal mirror symmetry-strong empirical
        support for prototype-surface theory.
  \item The residual 1.6 \% failures highlight a limitation: mirroring
        enforces symmetry but cannot guarantee a \emph{useful} initial
        orientation.  Runtime monitors or annealing remain valuable
        safety nets.
\end{itemize}

\hrulefill
