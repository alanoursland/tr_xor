% !TeX root = ../main.tex
\section{Runtime Monitors Study}
\label{sec:relu1-monitors}

\subsection*{Aim}
Rather than rejecting bad initialisations, we attach two \emph{online
monitors} that watch training in real time:

\begin{description}[leftmargin=2em,style=sameline]
  \item[DeadSampleMonitor] flags any input that is both misclassified
        and receives \emph{zero} gradient flow for more than five epochs,
        then nudges the closest hyperplane toward that sample.
  \item[BoundsMonitor] keeps every hyperplane within a radius
        $r = 1.4$ of the data mean; if a boundary drifts outside, its
        bias is reset to pass through the origin.
\end{description}

Early-stopping by "loss change $\!\!<\!10^{-24}$" is \emph{disabled}
so the monitors may act throughout all $800$ training epochs.
We ran \textbf{500} independent seeds to obtain a tight estimate of
reliability.

% ------------------------------------------------------------------
\subsection*{Classification Accuracy}

\begin{table}[h]
\centering
\caption{Final accuracy with runtime monitors ($500$ runs).}
\label{tab:relu1-monitor-accuracy}
\begin{tabular}{lccccc}
\toprule
Accuracy & 0\,\% & 25\,\% & 50\,\% & 75\,\% & 100\,\% \\
\midrule
Runs & 0 & 0 & 0 & 4 & 496 \\
\bottomrule
\end{tabular}
\end{table}

Success rises to \textbf{99.2\,\%}, matching the re-init\,+\,margin
strategy but \emph{during} training rather than before it.

% ------------------------------------------------------------------
\subsection*{Convergence Timing}

\begin{table}[h]
\centering
\caption{Epochs to $\mathcal L<10^{-7}$ (successful runs).}
\label{tab:relu1-monitor-epochs}
\begin{tabular}{lccccc}
\toprule
Percentile & 0\,\% & 25\,\% & 50\,\% & 75\,\% & 100\,\% \\ \midrule
Epochs & 49 & 133 & 160 & 192 & 800 \\
\bottomrule
\end{tabular}
\end{table}

Median time is comparable to the baseline; the long tail reflects runs
that linger near the loss threshold while the monitors make repeated
corrections.

% ------------------------------------------------------------------
\subsection*{Prototype-Surface Geometry}

\begin{description}[leftmargin=2em]
  \item[Distance clusters]
        992 hyperplanes fall into \textbf{two} patterns; the dominant
        one ($990$ members) matches $(d_{0},d_{1})\!=\!(0.10,1.41)$,
        indicating the surface anchors close to the False points while
        retaining the expected $\sqrt2$ gap to the True points.
  \item[Weight clusters]
        DBSCAN ($\varepsilon=0.1$) finds \textbf{two} sign-symmetric
        weight clusters with only four noise points—
        a tighter grouping than any previous method.
  \item[Mirror symmetry]
        Mirror pairs are detected in $487/500$ runs; $238$ are
        \emph{perfect} (cosine $\approx-1$).
\end{description}

Thus the monitors do not disturb the prototype geometry; if anything,
they strengthen the expected mirror structure.

% ------------------------------------------------------------------
\subsection*{Dead-Data Recovery}
Despite beginning with \textbf{dead inputs} in 360 runs, the monitors
revived almost all of them:

\begin{itemize}
  \item $360$ / $364$ runs with dead inputs ultimately reached
        100\,\% accuracy,
  \item only $4$ such runs stalled at 75\,\%.
\end{itemize}

% ------------------------------------------------------------------
\paragraph{Discussion}
\begin{itemize}
  \item Runtime correction achieves the same reliability as
        margin-based re-initialisation \emph{without} repeated weight
        sampling, at the expense of longer training time.
  \item Prototype-surface theory is \emph{reinforced}: a single distance
        pattern and two mirror weight clusters dominate.
\end{itemize}

\hrulefill
