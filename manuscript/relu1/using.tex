% relu1/using.tex
\section{Using This Chapter}
\label{sec:relu1_using}

This chapter can be read in two different modes:

\begin{description}
  \item[Survey\,/\,quick-skim]
        \begin{enumerate}
          \item Jump to \textbf{Sec.~\ref{sec:relu1_motivation}} for the one-page conceptual recap of why we revisit XOR with two unconstrained ReLU gates.
          \item Glance at the headline numbers and geometry snapshots in \textbf{Sec.~\ref{sec:relu1_baseline}} (the Kaiming baseline) to see \emph{why} the model needs help.
          \item Skip directly to the bullet \emph{Take-aways} in \textbf{Sec.~\ref{sec:relu1_takeaways}} for a concise list of what worked and why.
        \end{enumerate}

  \item[Deep-dive]
        \begin{enumerate}
          \item Read Sections \ref{sec:relu1_model_data} and \ref{sec:relu1_framework} first.  
                They repeat the centred-XOR dataset and experimental protocol from \emph{Chapter~Abs1} so you do \textbf{not} need to flip back, but feel free to skim if you remember the details.
          \item Work through the studies in the order they appear:
                \begin{itemize}
                    \item \textbf{Baseline} (\ref{sec:relu1_baseline})  
                          — establishes failure modes.
                    \item \textbf{Activation survey} (\ref{sec:relu1_activation})  
                          — Leaky/ELU/PReLU variants.
                    \item \textbf{Re-initialisation tactics} (\ref{sec:relu1_reinit})  
                          — simple vs.\ margin-based dead-data restarts.
                    \item \textbf{Bounded-hypersphere init} (\ref{sec:relu1_bhs})  
                          — geometry-aware weight sampling.
                    \item \textbf{Runtime monitors} (\ref{sec:relu1_runtime})  
                          — early detection of dead data or runaway weights.
                    \item \textbf{Loss-entropy annealing} (\ref{sec:relu1_anneal})  
                          — noise injection rescue.
                    \item \textbf{Mirror init} (\ref{sec:relu1_mirror})  
                          — hard-wiring symmetry.
                \end{itemize}
          \item Use the shaded “Result” boxes in each study for at-a-glance statistics; full plots and tables are in the accompanying figure panels.
        \end{enumerate}
\end{description}

\medskip\noindent
\textit{Notation}: We adopt the same symbol conventions as in \emph{Chapter~Abs1}.  Any parameters not re-defined here are identical to those earlier definitions.
